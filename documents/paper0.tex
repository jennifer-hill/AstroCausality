% This file is part of the AstroCausality project.
% Copyright 2014 David W. Hogg and Jennifer L. Hill.

\documentclass[12pt, letterpaper, preprint]{aastex}

\begin{document}

\begin{abstract}
The field of galaxy evolution is strongly data-driven; many important
results have emerged from statistical analyses of measurements of
galaxy samples.
Some of these results are presented in causal terms, for example, the
claim that star formation is shut off by processes that occur when a
galaxy falls into a high-density region.
In social-science disciplines, the attempt to discover causal
relations among quantities observed in samples of subjects is known as
``causal inference''.
Here we apply some of the nomenclature and methodology of
causal inference to galaxy data.
In our primary example, we consider the star-formation rate of a
galaxy as the ``outcome'', the large-scale galaxy environment as the
``treatment'', and the other galaxy properties (mass, velocity
dispersion, redshift, and so on) as ``confounders''.
To frame the question causally, we consider counterfactual states for
the galaxies; this permits precise definition of both what we are
estimating and the assumptions required to identify that quantity.
In this example we aim to reduce dependence on parametric assumptions
by using very flexible models.
We find XXX and YYY.
\end{abstract}

\section{Introduction}

Some comments about galaxy evolution.

Some comments about environment and star formation.

Make some comments on the relationship between causal inference and
conterfactual claims and how this relates to the (presumably)
deterministic dynamical evolution of the Universe.

\section{Data}

\section{Method}

\section{Discussion}

\end{document}
