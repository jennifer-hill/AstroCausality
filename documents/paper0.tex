% This file is part of the AstroCausality project.
% Copyright 2014 David W. Hogg.

\documentclass[12pt, letterpaper]{article}

\begin{document}

\begin{abstract}
The field of galaxy evolution is strongly data-driven; many important
results have emerged from statistical analyses of measurements of
galaxy samples.
Some of these results are presented in causal terms, for example, the
claim that star formation terminates when a galaxy falls into a
high-density region.
In social-science and medical disciplines, the attempt to discover
causal relations among quantities observed or measured in samples of
organizations or people is known as ``causal inference''.
Here we apply some of the the nomenclature and methodology of the
causal-inference literature to galaxy data.
In our primary example, we consider the star-formation rate of a
galaxy as the ``outcome'', the large-scale galaxy environment as the
``treatment'', and the other galaxy properties (mass, velocity
dispersion, redshift, and so on) as ``confounders''.
Causal inference proceeds by ZZZ.
We find XXX and YYY.
We make some comments on the relationship between causal inference and
conterfactual claims and how this relates to the (presumably)
deterministic dynamical evolution of the Universe.
\end{abstract}

\end{document}
