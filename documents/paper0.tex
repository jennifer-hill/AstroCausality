% This file is part of the AstroCausality project.
% Copyright 2014 David W. Hogg and Jennifer L. Hill.

\documentclass[12pt, letterpaper, preprint]{aastex}

\begin{document}

\begin{abstract}
The field of galaxy evolution is strongly data-driven; many important
results have emerged from statistical analyses of measurements of
galaxy samples.
Some of these results are presented in causal terms, for example, the
claim that star formation is shut off by processes that occur when a
galaxy falls into a high-density region.
In social-science disciplines, the attempt to discover causal
relations among quantities observed in samples of subjects is known as
``causal inference''.
Here we apply some of the nomenclature and methodology of
causal inference to galaxy data.
In our primary example, we consider the star-formation rate of a
galaxy as the ``outcome'', the large-scale galaxy environment as the
``treatment'', and the other galaxy properties (mass, velocity
dispersion, redshift, and so on) as ``confounders''.
To frame the question causally, we consider counterfactual states for
the galaxies; this permits precise definition of both what we are
estimating and the assumptions required to identify that quantity.
In this example we aim to reduce dependence on parametric assumptions
by using very flexible models.
We find XXX and YYY.
\end{abstract}

\section{Introduction}

One of the key empirical results of galaxy astrophysics is that
galaxies in higher density regions are---on average---older
and forming fewer stars than galaxies in lower-density regions (CITE).
This result has been confirmed with many different proxies for stellar
age and star-formation rate, and with many different proxies for
local density (CITE); it shows that the internal properties of galaxies
(related to star formation) are covariant with, or can be predicted
with, external properties (related to environmental density).
Of course, galaxies also have complex and informative relationships
among their purely internal properties:
Galaxy mass, spiral morphology, disk fraction, stellar radial profile,
and central stellar density are also show strong empirical
relationships with age and star formation rate (CITE).

Naturally, the scientific question has arisen, repeatedly (CITE):
What sets the star-formation rates of galaxies?
Does the entrance of a galaxy into a high density environment shut off
the star formation?
Or are the star-formation rates set by internal galaxy properties,
which themselves are differently represented in high density and low
density envirotnments?
These, fundamentally, are \emph{causal} questions, and importantly,
observed correlation, by itself, does not imply causation.

Here we look at these questions from the perspective of \emph{causal
  inference}, the branch of statistics that develops the assumptions
and techniques required to make a causal claim based on empirical
evidence.
Causal inference is employed often in the social sciences and health
sciences where, often (and always in astronomy) the data are
``observed'' in natural experiments rather than taken in designed and
controlled experiments.
The strongest causal results have been obtained in randomized trials,
in which a prediction about a causal effect has been made, subjects
have been randomly assigned to two groups (a ``treatment group'' and a
``control group''), treatment has been applied and outcomes assessed.
In randomized experiments, the causal hypothesis makes a clear prediction
that can be directly tested.
In astronomy, we are not permitted to perform such randomized trials.
Their execution would require moving galaxies from one environment to
another and looking for changes in star-formation rates subsequent to
the moves.
Experiments of this nature would require giant actors or machines
(galaxy-moving machines) working at cosmological scale, and more
patience than the astronomical community can muster, even taken as a
whole.
(Not to mention the environmental impact assessment requirements.)

This ``giant actors'' fantasy is useful for bringing out ideas about
causal inference in observational data.
For one, it illustrates that we will have to make counterfactual
hypotheses:
We will have to consider what \emph{would have happened} to a galaxy
\emph{if it had been} in a different environment.
We can't move the galaxy to the different environment, so we will have
to build a model that can predict that outcome, and infer its
parameters using the data.

For another, it indicates---philosophically---that there is something
ill-posed about the concept of causality in a deterministic universe.
Galaxies form, evolve, and move according to (we think) deterministic
laws from initial conditions; at the microscopic level these laws are
(we think) time-reversible.
There is no really plausible counterfactual state for the galaxies we
observe; if we wanted to self-consistently swap the locations of
galaxy A and galaxy B, we would have to re-run the entire time history
of the Universe with different initial conditions.
Once we re-run everything from scratch, the concept of isolated causal
``treatments''---environment vs galaxy mass, for instance---is not
really coherent.
In this sense, it might be that macroscopic causal questions in the
Universe such as those asked in the galaxy evolution literature (by
CITE, just to name a few) are not really well posed at all.
That is one possible conclusion of this work:
\emph{Stop asking causal questions!}
If you have read this far, and still want to know whether galaxy
star-formation rates are set by the environment external to the galaxy
or just by internal properties of the galaxy itself, then you may
proceed below this line, provided that you recognize that you might be
asking some kind of non-question.
Don't feel bad:
Most questions of great interest in the history of science have indeed
been framed causally.

When observational data---data from a natural experiment over which
the observers have little or no control---are used to infer a causal
relationship, there are three kinds of serious issues:
First, it is imperative that the data and the analysis include good
information about all important \emph{confounders}.
[HOGG: Give example here for counfounders.]
Second, it is important to separate pre-treatment from post-treatment
variables.
[HOGG: Continue the example here.]
Third, it is key that the subjects of the study be independent, or
non-interfering.
[HOGG: And some more.]

We hope it is obvious to the reader that all three of these problems
are substantial in the galaxy astrophysics context:
We do not have a full mechanical model for star-formation activity in
galaxies, so we do not have a clear understanding of all the possible
confounders.
(Indeed, if we \emph{did} have a full mechanical model for
star-formation activity, we wouldn't be asking these ill-posed
questions about causality!)
We cannot separate pre-treatment from post-treatment effects with any
degree of certainty.
Indeed, in a deterministic Universe, it is not obvious that it is even
possible to make this distinction.
We don't think that the galaxies---especially those in high density
environments---are truly independent and non-interfering.
On that latter point, we might be able to get away with
sparse-sampling the close pairs.

\section{Causality generalities}

There is an outcome, a treatment, and many confounders.

Causal inference is challenging for many reasons, not limited to the following:
The confounders affect the outcome.
The treatment is probably not given to the population as it would be in a randomized trial.
The confounders that affect the outcome might also affect whether the treatment is given.
The outcome might cause the treatment!  Or worse!
And so on.

We treat the causal inference problem like a missing-data problem:
We don't get to observe the outcomes that would have happened if the untreated had been treated,
and if the treated had not been treated;
we must infer these unobserved---or potential---outcomes.

Here is a toy example of some kind.

\section{Data}

\section{Method}

\section{Discussion}

\end{document}
