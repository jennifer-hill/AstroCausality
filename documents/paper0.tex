% This file is part of the AstroCausality project.
% Copyright 2014 David W. Hogg.

\documentclass[12pt, letterpaper, preprint]{aastex}

\begin{document}

\begin{abstract}
The field of galaxy evolution is strongly data-driven; many important
results have emerged from statistical analyses of measurements of
galaxy samples.
Some of these results are presented in causal terms, for example, the
claim that star formation is shut off by processes that occur when a
galaxy falls into a high-density region.
In social-science disciplines, the attempt to discover causal
relations among quantities observed in samples of subjects is known as
``causal inference''.
Here we apply some of the nomenclature and methodology of
causal inference to galaxy data.
In our primary example, we consider the star-formation rate of a
galaxy as the ``outcome'', the large-scale galaxy environment as the
``treatment'', and the other galaxy properties (mass, velocity
dispersion, redshift, and so on) as ``confounders''.
Causal inference proceeds by very flexibly modeling the relationship
between the outcome and the confounders and then asking if the
outcomes for subjects similar under the confounders are likely to have
been different under different treatments.
We find XXX and YYY.
We make some comments on the relationship between causal inference and
conterfactual claims and how this relates to the (presumably)
deterministic dynamical evolution of the Universe.
\end{abstract}

Hello World

\end{document}
